\documentclass{article}
\usepackage[utf8]{inputenc}

\title{\textbf{Comentario Crítico (Astrocitos)} \\ \bigskip \large Grupo 1.5 (Lunes, 12:30 - 14:30)}

\author{David Rodríguez Bacelar, Sergio Rodríguez Seoane y Anxo Portela Iglesias}
\date{19 de Abril de 2021}

\begin{document}

\maketitle

Para entender el primer artículo y el contexto que rodea a las células gliales -y en particular a los
astrocitos-, es necesario profundizar primero en el paradigma neurológico actual, en los elementos que lo 
componen y las novedosas interacciones que se dan entre ellos.
\bigskip

\noindent
Antes de la década de los 90, como se remarca en todos los artículos, era sabido el hecho de que nuestro cerebro
estaba compuesto por dos tipos de células: las neuronas y las células gliales. Las primeras eran eléctricamente
excitables y, ya que la actividad nerviosa tenía una base eléctrica, serían éstas las encargadas de transportar
y transformar la información. Las segundas, en cambio, se pensaba que solo servían como células de apoyo a las neuronas,
proporcionándoles nutrientes y un sustento estructural.
\bigskip

\noindent
Pero, a partir de la aparición de instrumentos de medición más sofisticados que permitían, entre otras cosas, el análisis
cerebral \textit{in vivo}, es cuando se detecta que existe una variación en los niveles de calcio intracelulares
de los astrocitos en un proceso sináptico. Es decir, que los astrocitos participaban de la comunicación entre neuronas.
\bigskip

\noindent
Surge así un nuevo fenómeno: la sinapsis tripartita. En ella, las terminales sinápticas de las neuronas
podrían liberar neurotransmisores químicos que, dada la proximidad entre neuronas y astrocitos, pudieran activar los receptores
de estos últimos, desencadenando las señales de calcio antes mencionadas. Cabe destacar tambien que los astrocitos no solo serían meros receptores
en la sinapsis, sino que también podrían producir neurotransmisores que influyesen en la comunicación neuronal dependiendo de las moléculas
recibidas.

\newpage
Este avance en el terreno de la neurología, podría ofercernos una hipótesis sobre el porqué de la superioridad
intelectual de la especie humana. Distintos análisis han evidenciado como, a pesar de poseer una masa crebral 3 veces mayor
que los chimpacés, la diferencia entre el número de neuronas es apenas de un 25\%. Esto, junto a que es en el cerebro humano donde
la proporción astrocito-neurona es mayor y a los resultados del experimento de Goldman en ratas del primer artículo,
nos da una pista de la importancia de las células gliales en el procesamiento de la información.


\bigskip

\noindent
Estos nuevos descubrimientos hacen que los astrocitos pasen a ser elementos fundamentales en el sistema nervioso, lo que supone un cambio en el paradigma actual
en campos como la medicina. Hasta ahora, el tratamiento de enfermedades cerebrales se focalizaba únicamente en las neuronas; este nuevo enfoque podría ayudar a 
ofrecer nuevas alternativas que tengan en cuenta el papel de ambos tipos de células. Enfermedades que afectan a las capacidades de aprendizaje podrían verse mejoradas
con el implante de astrocitos y los sítomas de enfermedades neurodegenerativas como el alzhéimer podrían verse reducidos.
\bigskip

\noindent
La comprensión de las interacciones de los astrocitos en los procesos sinápticos ofrece también nuevos caminos de mejora en disciplinas como la inteligencia artificial,
especialmente en las redes neuronales artificiales. Ya que éstas se basan en el funcionamiento biológico real de nuestro cerebro, la inclusión de los astrocitos podría
aportar una mejora en el rendimiento y ofrecer mejores resultados. Los astrocitos se simularían como un contador que se situaría en las conexiones entre las 
neuronas artificiales y que regularía la comunicación entre ellas.
\bigskip

\noindent
En conclusión, descubrimientos como estos abren nuevos horizontes en el mundo de la investigación y pueden ser la base de nuevos avances en la neurología.
Abren nuevas puertas que pueden mejorar nuestras vidas, aportando soluciones a grandes problemas como las enfermedades cerebrales o permitiendo avances 
en conocimientos punteros hoy en día como la IA.


\end{document}
